\documentclass[diplomskirad, numeric, utf8, times]{fer}
% Dodaj opciju upload za generiranje konačne verzije koja se učitava na FERWeb
% Add the option upload to generate the final version which is uploaded to FERWeb


\usepackage{blindtext}
\usepackage{subfiles}
\usepackage{ragged2e}
\usepackage[dvipsnames]{xcolor}

\usepackage{courier}
\usepackage{booktabs}
\usepackage{import}
\usepackage{subfiles}
\usepackage{amsmath}
\usepackage{url}
\usepackage{graphicx}
\usepackage[utf8]{inputenc}
\usepackage[T1]{fontenc}
\usepackage{ragged2e}

\usepackage{xcolor}
\usepackage{listings}
%\usepackage[nottoc]{tocbibind}

\usepackage{float}
\floatstyle{plaintop}
\restylefloat{table}

\definecolor{mGreen}{rgb}{0,0.6,0}
\definecolor{mGray}{rgb}{0.5,0.5,0.5}
\definecolor{mPurple}{rgb}{0.58,0,0.82}
\definecolor{backgroundColour}{rgb}{0.95,0.95,0.92}

\lstdefinestyle{CStyle}{
    backgroundcolor=\color{backgroundColour},   
    commentstyle=\color{mGreen},
    keywordstyle=\color{magenta},
    numberstyle=\tiny\color{mGray},
    stringstyle=\color{mPurple},
    columns=fixed,
    basicstyle=\ttfamily,
    breakatwhitespace=false,         
    breaklines=true,                 
    captionpos=b,                    
    keepspaces=true,                 
    numbers=left,                    
    numbersep=5pt,                  
    showspaces=false,                
    showstringspaces=false,
    showtabs=false,                  
    tabsize=4,
    language=C
}

%\renewcommand\thelstlisting{\ifnum \c@chapter>\z@ \thechapter\@\fi \@arabic\c@lstlisting.}


\renewcommand{\lstlistingname}{Odsječak koda}
\renewcommand{\figurename}{Slika}
\renewcommand{\tablename}{Tablica}
\renewcommand\bibname{Literatura}

%--- PODACI O RADU / THESIS INFORMATION ----------------------------------------

% Naslov na engleskom jeziku / Title in English
\title{Software for a user-centric wearable wireless textile system for monitoring body fluids based on a Zephyr operating system}

% Naslov na hrvatskom jeziku / Title in Croatian
\naslov{Programska potpora za korisnički usmjereni tekstilni bežični nosivi sustav za praćenje tjelesnih tekućina temeljena na operacijskom sustavu Zephyr}

% Broj rada / Thesis number
\brojrada{1234}

% Autor / Author
\author{Luka Jengić}

% Mentor 
\mentor{Prof.\@ Hrvoje Džapo}

% Datum rada na engleskom jeziku / Date in English
\date{June, 2024}

% Datum rada na hrvatskom jeziku / Date in Croatian
\datum{lipanj, 2024.}

%-------------------------------------------------------------------------------

\begin{document}

% Naslovnica se automatski generira / Titlepage is automatically generated
\maketitle


%--- ZADATAK / THESIS ASSIGNMENT -----------------------------------------------

% Zadatak se ubacuje iz vanjske datoteke / Thesis assignment is included from external file
% Upiši ime PDF datoteke preuzete s FERWeb-a / Enter the filename of the PDF downloaded from FERWeb
\zadatak{Luka Jengić - tekst zadatka.pdf}


%--- ZAHVALE / ACKNOWLEDGMENT --------------------------------------------------

\begin{zahvale}
  % Ovdje upišite zahvale / Write in the acknowledgment
  Hvala na rakiji, popij i ti jednu...
\end{zahvale}


% Odovud započinje numeriranje stranica / Page numbering starts from here
\mainmatter


% Sadržaj se automatski generira / Table of contents is automatically generated
\tableofcontents


%--- UVOD / INTRODUCTION -------------------------------------------------------
\chapter{Uvod}
\label{pog:uvod}

\subfile{chepters/uvod}

%-------------------------------------------------------------------------------
\chapter{Analiza sastava ljudskog tijela}
\label{pog:glavni_dio}

\subfile{chepters/bioimpedancija}

%-------------------------------------------------------------------------------
\chapter{Metode mjerenja bioimpedancije}

\subfile{chepters/mjerenje_bioimpedancije}

%-------------------------------------------------------------------------------
\chapter{Razvijeni nosivi sustav za praćenje tjelesnih tekućina}

\subfile{chepters/razvijeni_sustav}

%-------------------------------------------------------------------------------
\chapter{Programska podrška za razvijeni sustav}
\label{chap:programska_podrska}

\subfile{chepters/programska_potpora}

%-------------------------------------------------------------------------------
\chapter{Razvijeno ispitno okruženje}

\subfile{chepters/ispitno_okruzenje}

%-------------------------------------------------------------------------------
\chapter{Laboratorijska mjerenje}

\subfile{chepters/laboratorijska_mjerenja}

%-------------------------------------------------------------------------------
\chapter{Rezultati i diskusija}

\subfile{chepters/rezultati}

%--- ZAKLJUČAK / CONCLUSION ----------------------------------------------------
\chapter{Zaključak}
\label{pog:zakljucak}

\subfile{chepters/zakljucak}


%--- LITERATURA / REFERENCES ---------------------------------------------------

% Literatura se automatski generira iz zadane .bib datoteke / References are automatically generated from the supplied .bib file
% Upiši ime BibTeX datoteke bez .bib nastavka / Enter the name of the BibTeX file without .bib extension
\nocite{*}
\bibliography{literatura}


%--- SAŽETAK / ABSTRACT --------------------------------------------------------

% Sažetak na hrvatskom
\begin{sazetak}

Ovaj rad istražuje različite mjerne metode i senzorske sustave namijenjene 
praćenju tekućine u nogama i određivanju mase mišića listova.
U sklopu rada razvijen je nosivi sustav temeljen na MAX30009 integriranom sučelju za mjerenje bioimpedancije. 
Razvijena je programska potpora za rad u stvarnom vremenu temeljena na operacijskom sustavu Zephyr. 
Implementirana je funkcionalnost za prikupljanje i obradu podataka sa senzora te je razvijen 
protokol za bežičnu komunikaciju s drugim sustavima putem Bluetooth Low Energy komunikacijskog protokola.
Dodatno, razvijeno je ispitno okruženje za testiranje sustava u stvarnim uvjetima korištenja, 
što je omogućilo provođenje laboratorijskih mjerenja, ispitivanje značajki razvijenog programskog sustava 
te vizualizaciju dobivenih rezultata. 
Velika važnost stavljena je na korisničko iskustvo, osiguravajući da je sustav intuitivan i 
jednostavan za korištenje krajnjim korisnicima.
Laboratorijska mjerenja provedena su s tekstilnim i gel elektrodama te je razvijeni sustav uspoređen 
s referentnim sustavom za mjerenje bioimpedancije SFB7 ImpediMed.

\end{sazetak}

\begin{kljucnerijeci}
nosivi sustavi, sastav ljudskog tijela, bioimpedancija, MAX30009, Bluetooth Low Energy, Zephyr 
\end{kljucnerijeci}


% Abstract in English
\begin{abstract}
This paper explores various measurement methods and sensor systems designed for monitoring fluid 
in the legs and determining calf muscle mass. 
As part of the study, a wearable system based on the MAX30009 integrated bioimpedance 
measurement interface was developed. 
Real-time software support was implemented using the Zephyr operating system, 
enabling data collection and processing from the sensors. 
A protocol for wireless communication using Bluetooth Low Energy 
was also developed to interface with other systems.
Additionally, a test environment was created to evaluate the system under real-world conditions, 
facilitating laboratory measurements, testing of software features, 
and visualization of results. Emphasis was placed on user experience, 
ensuring the system is intuitive and user-friendly. 
Laboratory measurements were conducted using textile and gel electrodes, 
and the developed system was compared with the reference bioimpedance measurement system, SFB7 ImpediMed.
\end{abstract}

\begin{keywords}
wearable devices, body composition, bioimpedance, MAX30009, Bluetooth Low Energy, Zephyr 
\end{keywords}


%--- PRIVITCI / APPENDIX -------------------------------------------------------

% Sva poglavlja koja slijede će biti označena slovom i riječi privitak / All following chapters will be denoted with an appendix and a letter
\backmatter

%\chapter{The Code}

%\Blindtext


\end{document}
