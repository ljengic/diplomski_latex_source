\documentclass[../diplomski_rad.tex]{subfiles}

\begin{document}

\sloppy

\justifying

U ovom radu razvijena je programska potpora za bežični nosivi sustav za praćenje distribucije tjelesnih tekućina u donjim ekstremitetima
ispitanika mjerenjem bioimpedancije. 
Sustav je namijenjen za kontinuirano praćenje stanja pacijenta koji boluju od zatajenja srca te je prikladan 
za primjenu i nošenje na ispitaniku u svakodnevnom okruženju. 
Sustav se temelji na STM32WB5MMG bežičnom modulu te MAX30009 integriranom sučelju za mjerenje bioimpedancije. 
Programska potpora za razvijeni sustav temeljena je na operacijskom sustavu Zephyr.  
Dodatno, razvijeno je ispitno okruženje
za testiranje sustava u stvarnim uvjetima korištenja. 
Ispitno okruženje razvijeno je u obliku desktop aplikacije u programskom jeziku Python.
Velika važnost posvećena je korisničkom iskustvu, osiguravajući da je sustav intuitivan i 
jednostavan za korištenje krajnjim korisnicima.

% zakljucak je narativan bez natuknica pa sam to popravio
Provedena su laboratorijska mjerenja na šest ispitanika kako bi se validirala funkcionalnost sustava. Analizom izmjerenih podataka prikupljenih tijekom eksperimenata vježbe plantarne fleksije zaključeno je da se razvijeni sustav može koristiti za kontinuirano praćenje promjena tekućine tijekom vježbanja plantarnih fleksija te da se tijekom intenzivne aktivnosti redistribucija tekućine iz vaskularnog prostora u intersticijalni prostor može detektirati kao promjena u otporu i reaktanciji. Stoga ova tehnika može pružiti uvid u hidriranost mišića i biti korisna za procjenu stanja sportaša, planiranje rehabilitacije i optimizaciju performansi.