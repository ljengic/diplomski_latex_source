\documentclass[../diplomski_rad.tex]{subfiles}

\begin{document}

\sloppy

\justifying

U ovom radu razvijena je programska potpora za bežični nosivi sustav za praćenje distribucije tjelesnih tekućina u donjim ekstremitetima
ispitanika mjerenjem bioimpedancije. 
Sustav je namijenjen za kontinuirano praćenje stanja pacijenta koji boluju od zatajenja srca te je prikladan 
za primjenu i nošenje na ispitaniku u svakodnevnom okruženju. 
Sustav se temelji na STM32WB5MMG bežičnom modulu te MAX30009 integriranom sučelju za mjerenje bioimpedancije. 
Programska potpora za razvijeni sustav pisana je s pomoću operacijskog sustava Zephyr.  
Dodatno, razvijeno je ispitno okruženje
za testiranje sustava u stvarnim uvjetima korištenja. 
Ispitno okruženje razvijeno je u obliku desktop aplikacije u programskom jeziku Python.
Velika važnost posvećena je korisničkom iskustvu, osiguravajući da je sustav intuitivan i 
jednostavan za korištenje krajnjim korisnicima.

Provedena su laboratorijska mjerenja na šest ispitanika kako bi se validirala funkcionalnost sustava. 
Analizom izmjerenih podataka prikupljenih tijekom eksperimenata vježbe plantarne fleksije doneseni su sljedeći zaključci: 
\begin{itemize}
    \item razvijeni sustav može se koristiti za kontinuirano praćenje promjena tekućine tijekom vježbanja plantarnih fleksija,  
    \item tijekom intenzivne aktivnosti, redistribucija tekućine iz vaskularnog prostora u intersticijalni prostor 
    može se detektirati kao promjene u otporu i reaktanciji,
    \item ova tehnika može pružiti uvid u hidriranost mišića i može biti korisna za procjenu stanja sportaša, 
    planiranje rehabilitacije i optimizaciju performansi. 
\end{itemize} 

\end{document}