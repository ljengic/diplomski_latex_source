\documentclass[../diplomski_rad.tex]{subfiles}

\begin{document}

\sloppy

\justifying

Rezultati pružaju uvid u varijabilnost bioimpedancijskih mjerenja tijekom vježbanja i mogu se 
koristiti za daljnja istraživanja biomehaničkih i fizioloških odgovora na tjelesnu aktivnost. 
Bioimpedancija se pokazuje kao vrijedna metoda za praćenje fizioloških promjena u realnom vremenu, 
pružajući kvantitativne podatke o stanju mišića i tekućine tijekom vježbanja.

Metoda bioimpedancije može se koristiti za praćenje promjena tekućine tijekom vježbanja plantarnih fleksija. 
Tijekom intenzivne aktivnosti, redistribucija tekućine iz vaskularnog prostora u intersticijalni prostor 
može se detektirati kao promjene u otporu i reaktanciji. 
Ova tehnika može pružiti uvid u hidriranost mišića i može biti korisna za procjenu stanja sportaša, 
planiranje rehabilitacije i optimizaciju performansi.


\end{document}