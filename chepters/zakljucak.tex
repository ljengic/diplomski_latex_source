\documentclass[../diplomski_rad.tex]{subfiles}

\begin{document}

\sloppy

\justifying

U ovom radu razvijen je i testiran bežični nosivi sustav za praćenje distribucije tjelesnih tekućina u donjim ekstremitetima
ispitanika mjerenjem bioimpedancije. Sustav je prikladan za primjenu i nošenje na ispitaniku u svakodnevnom okruženju. Provedena su laboratorijska mjerenja na šest ispitanika kako bi se validirala funkcionalnost sustava. 
Analizom izmjerenih podataka prikupljenih tijekom eksperimenata doneseni su sljedeći zaključci.

Rezultati pružaju uvid u varijabilnost bioimpedancijskih mjerenja tijekom vježbanja i mogu se 
koristiti za daljnja istraživanja biomehaničkih i fizioloških odgovora na tjelesnu aktivnost. 
Bioimpedancija se pokazuje kao vrijedna metoda za praćenje fizioloških promjena u stvarnom vremenu, 
pružajući kvantitativne podatke o stanju mišića i tekućine tijekom vježbanja.

Metoda bioimpedancije može se koristiti za praćenje promjena tekućine tijekom vježbanja plantarnih fleksija. 
Tijekom intenzivne aktivnosti, redistribucija tekućine iz vaskularnog prostora u intersticijalni prostor 
može se detektirati kao promjene u otporu i reaktanciji. 
Ova tehnika može pružiti uvid u hidriranost mišića i može biti korisna za procjenu stanja sportaša, 
planiranje rehabilitacije i optimizaciju performansi.

\end{document}