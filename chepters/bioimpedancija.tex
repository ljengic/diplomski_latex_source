\documentclass[../diplomski_rad.tex]{subfiles}

\begin{document}

\sloppy

\justifying

%uvod u poglavlje
Analiza sastava ljudskog tijela je proces procjene udjela različitih tjelesnih komponenti 
poput masti, mišića i tekućina.
Dobiveni rezultati pružaju važne informacije koje se koriste u praćenju zdravlja, 
procjeni rizika od pojedinih bolesti, praćenju oporavka te ranom otkrivanju zdravstvenih problema.

Mjerenje bioimpedancije tijela jedna je od metoda kojom se procjenjuje sastav ljudskog tijela. 
Kroz tijelo se pušta slaba struja, reda veličine mikroampera, te se mjeri pad napona čime se izračunava impedancija tijela. 
Mjerenjem bioimpedancije moguće je praćenje distribucije tekućina u tijelu. 
Praćenje distribucije tekućine kroz tijelo je vrijedna dijagnostička metoda 
za praćenje razvoja srčanih bolesti \cite{Abassi2022}.

\section{Sastav ljudskog tijela}

Ljudsko tijelo je kompleksna biološka struktura koja se sastoji od različitih međusobno povezanih tkiva koja 
omogućavaju funkcioniranje organizma. Približno se sastoji od 
64\% vode,
20\% proteina,
10\% masti 
i 5\% minerala.
Važno je napomenuti kako sastav ljudskog tijela varira od osobe do osobe jer na njega utječu 
pojedini faktori, kao što su spol i dob \cite{Bera2014}.  

\begin{figure}[htb]
    \centering
    \includegraphics[width=0.5\textwidth]{Figures/sastav_tijela_2.png} 
    \caption{Udio vode, proteina, masti i minerala u ljudskom tijelu \cite{Bera2014}}
    \label{slk:sastav_tijela}
\end{figure}

Sastav ljudskog tijela prikazan je na slici \ref{slk:sastav_tijela}.
Voda je osnovni element stanica i tkiva te je nužna za brojne fiziološke procese u organizmu, 
kao na primjer održavanje elektrolitske ravnoteže i regulacija temperature.
Ukupnu vodu u tijelu (engl. \textit{Total Body Water; TBW}) 
dijelimo na intracelularnu vodu (engl. \textit{Intracellular Water; ICW}) i ekstracelularnu vodu (engl. \textit{Extracellular Water; ECW}). 
Važni parametri pri analizi ljudskog tijela su i masa tijela bez masnog tkiva (engl. \textit{Fat Free Mass; FFM}) 
te masa masnog tkiva (engl. \textit{Fat Mass; FM}) \cite{Bera2014}.

Ekstracelularna voda je količina vode koja se nalazi izvan stanica te čini 30-40\% ukupne vode. Uključuje krv, limfu, tekućinu u 
zglobovima i međustaničnom prostoru. Ima važnu ulogu u transportu kisika i hranjivih tvari do stanica te odvođenju otpadnih 
tvari iz organizma \cite{Bera2014}.

Intracelularna voda je voda koja se nalazi unutar citoplazme stanica, predstavljajući ključnu 
komponentu u održavanju stanične homeostaze i omogućavajući različite biokemijske reakcije 
koje su neophodne za životne procese.
Održavanje ravnoteže između ekstracelularne i intracelularne vode ključno je za normalno funkcioniranje organizma \cite{Bera2014}.

Masno tkivo je također važno za funkcioniranje organizma jer pruža energetsku rezervu, toplinsku izolaciju te štiti unutarnje organe. 
Prekomjerno nakupljanje masnoće može dovesti do različitih zdravstvenih problema, poput pretilosti, dijabetesa i 
bolesti kardiovaskularnog sustava. 
Zbog toga je praćenje udjela masnog tkiva u organizmu važno u procijeni rizika od brojnih bolesti \cite{Bera2014}.

Masu tijela bez masnog tkiva dobijemo tako da od ukupne mase tijela oduzmemo masu masnog tkiva. 
FFW uključuje tjelesnu vodu, mišiće, kosti, organe i druga tkiva osim masnih tkiva te predstavlja masu koja je aktivna i sudjeluje 
u metaboličkim procesima \cite{Bera2014}.

\begin{figure}[htb]
    \centering
    \includegraphics[width=0.85\textwidth]{Figures/sastav_tijela.png} 
    \caption{Sastav ljudskog tijela \cite{Bera2014}}
    \label{slk:sastav_tijela}
\end{figure}

Koliko dobro će tkivo provoditi struju, ovisi o količini vode u njemu. 
Tkiva koja imaju više vode u sebi, kao na primjer mišići, bolje provode eletričnu struju nego masno tkivo koje ne sadrži vodu. 
Zbog toga se sastav ljudskog tijela procjenjuje iz izmjerene bioimpedancije između različitih dijelova tijela. 
Iz bioimpedancije sastav tijela se dobiva putem teorijskih jednadžbi ili tablica koje ovise o parametrima 
kao što su spol, dobna skupina, težina i visina.

\section{Praćenje distribucije tekućine u nogama}

U ovom radu, naglasak je stavljen na praćenje distribucije tekućine u potkoljenici. 
Sastav potkoljenice može se dobiti mjerenjem njezine bioimpedancije te zatim korištenjem teorijskih jednadžbi. 
Prvi korak je određivanja volumena potkoljenice. 
Potkoljenica je aproksimirana cilindrom te se njezin volumen određuje pomoću jednadžbe \ref{jed:volmen_noge}, 
pri čemu je \textit{L} razmak izmešu naponskih elektroda, a \textit{O} opseg potkoljenice.

\begin{equation}
    \label{jed:volmen_noge}
    \mathrm{V}=L A=L \frac{O^2}{4 \pi}
\end{equation} 

Sljedeći korak je mjerenje bioimpedancije, postavljanjem elektroda na način prikazan na slici (UMETNI SLIKU). 
Vrijdnost koje su potrebne za daljnje izračune su $R_{0}$ i $R_{\infty}$.
$R_{0}$ i $R_{\infty}$ predstavljaju otpor tkiva na nultoj i beskonačnoj frekvenciji te će detaljno biti opisani u poglavlju \ref{chap:bioz}.

Za izračunavanje volumena TBW, ECW i ICW u potkoljenici korištene su formule opisane u \cite{Delano2022}:

\begin{equation}
    \label{jed:ecw_noge}
    ECW=\frac{1}{1000} *\left(\frac{\rho_e L^2 \sqrt{V}}{R_0}\right)^{\frac{2}{3}}
\end{equation} 
\begin{equation}
    \label{jed:tbw_noge}
    TBW=\frac{1}{1000} *\left(\frac{\rho_{\infty} L^2 \sqrt{V}}{R_{\infty}}\right)^{\frac{2}{3}}
\end{equation} 
\begin{equation}
    \label{jed:icw_noge}
    ICW=TBW-ECW 
\end{equation} 

pri čemu je $\rho_{e}$ efektivna otpornost ekstracelularne tekućine iznosa 273,9 $\Omega$cm, a 
$\rho_{\infty}$ efektivna otpornost ukupne tjelesne tekućine iznosa 937,2 $\Omega$cm.

\section{Određivanje mase listova}

\end{document}