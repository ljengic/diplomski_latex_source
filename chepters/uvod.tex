\documentclass[../diplomski_rad.tex]{subfiles}

\begin{document}

\sloppy

\justifying

Globalna populacija starijih osoba rapidno raste, pri čemu se očekuje da će udio osoba 
starijih od 65 godina porasti s 10\% u 2022. na 16\% do 2050. godine \cite{Chen2023}. 
Ovaj demografski trend prema starenju populacije, uz istovremeni pad radne snage zdravstvenih djelatnika, predstavlja značajne izazove za 
javno zdravstvo i socioekonomske sustave. 
%Rastući broj starijih osoba, uz istovremeni pad radne snage skrbnika, stavlja značajan pritisak na 
%globalne zdravstvene sustave u smislu operativnih troškova i raspodjele resursa.
S porastom očekivane životne dobi povećava se i broj kroničnih bolesti poput dijabetesa, 
bolesti srca, artritisa te neurodegenerativnih bolesti poput Alzheimerove ili Parkinsonove bolesti. 
Upravljanje ovim bolestima zahtijeva kontinuirano praćenje i medicinsku skrb u čemu 
nosive tehnologije mogu donijeti značajan doprinos. 

Tradicionalan pristup zdravstvenoj procjeni oslanja se na posjet liječniku, što troši resurse zdravstvenog sustava 
i potencijalno može rezultirati prekasnom dijagnozom. 
Također, na taj način pacijentovo stanje ne može se pratiti kontinuirano kroz dulje vremensko razdoblje.
Primjena digitalnih tehnologija u kliničkoj praksi rezultirat će kvalitetnijom, kontinuiranom skrbi za pacijente kao i 
efikasnijim zdravstvenim sustavom. 
Nadalje, nosivi sustavi mogu generirati trenutne alarme u slučaju hitnih situacija poput moždanog udara, 
napadaja ili pada, omogućujući pravovremene medicinske intervencije \cite{Chen2023}.

U ovom radu fokus je stavljen na pacijente koji boluju od zatajenja srca.
Zatajenje srca je medicinsko stanje koje se javlja kada srce nije u mogućnosti pumpati dovoljno krvi kako bi 
zadovoljilo potrebe tijela. Jedan od uobičajenih simptoma zatajenja srca je periferna edema, odnosno nakupljanje 
tekućine u tkivima, posebno u donjim ekstremitetima \cite{Abassi2022}. Zbog toga je praćenje tjelesnih tekućina izuzetno važna 
dijagnostička metoda za procjenu stanja pacijenata koji boluju od zatajenja srca. Jedna od neinvazivnih metoda 
procjene volumena tjelesne tekućine je bioimpedancijska spektrografija. 

U sklopu diplomskom radu razvijen je nosivi sustav za praćenje tijelesnih tekućina nogu kao i ispitno okruženje za njegovo testiranje. 
Sustav se temelji na MAX30009 integriranom sučelju za mjerenje bioimpedancije i STM32WB55MMG bežičnom modulu koji omogućava komunikaciju 
sa razvijenim ispitnim okruženjem. Sustav je testiran i uspoređen sa referentnim uređajem za mjerenje bioimpedancije SFB7 ImpediMed.
Sustav je namijenjen starijoj populaciji te je stoga prilikom razvoja uređaja i ispitnog okruženja 
posebna pažnja pridana jednostavnosti i intuitivnosti korištenja. 

\end{document}