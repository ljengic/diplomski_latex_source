\documentclass[../diplomski_rad.tex]{subfiles}

\begin{document}


\sloppy

\justifying

U ovom poglavlju iznesen je proces razvoja programske potpore za ranije opisani nosivi ugradbeni sustav. 
Programska potpora za korišteni mikrokontroler STM32WB55VGY razvijena je pomoću operacijskog sustava 
za ugradbena računala Zephyr.

Za utakanje programske potpore na razvijenu pločicu i testiranje sustava korišten je ST-LINK/V2 programator \cite{stm32programator}. 
Progamator se na mikrokontroler spaja putem SWD sučelja (engl. \textit{Serial Wire Debug}). 
Kao razvojno okruženje korišten je Visual Studio Code, a kontrola verzija praćena je sustavom git.

Ovaj integrirani pristup omogućio je efikasno razvijanje i upravljanje programskom podrškom za STM32 mikrokontroler, 
uz korištenje popularnih alata i tehnologija. 
U nastavku će se detaljno opisati proces razvoja, implementacija, te testiranja programske potpore, 
uz naglasak na integraciju sa Zephyr operacijskim sustavom.

\section{Programiranje mrežnog procesora ARM Cortex-M0+}

STMicroelectronics pruža već gotove binarne datoteke \cite{kodovi_M0} koje sadrže kod komunikacijskog stoga. 
Postoje različite verzije komunikacijskog stoga s obzirom na primjenu te je samo potrebno pronaći odgovarajuću 
binarnu datoteku i učitati ju na jezgru ARM Cortex-M0+.

Za potrebe ovog projekta odabrana je datoteka \textit{stm32wb5x\_BLE\_HCILayer\_extended\_fw.bin} jer je kompatibilna sa 
korištenim operacijskim sustavom Zephyr. Datoteka je učitana na procesor pomoću programa STM32CubeProgrammer v2.15.0 
koji u sebi ima ugrađenu podršku za ažuriranje koda komunikacijskog stoga. 

\section{Operacijski sustav za rad u stvarnom vremenu Zephyr}

-sta je zepyhr i tako to

-dts opisati (al to na kraju kad dode hardware)

\section{Opis programske potpore za prikupljanje podataka sa pojedinih senzora}

\section{Bluetooth low energy komunikacija}

Bluetooth Low Energy (BLE) je bežični komunikacijski protokol koji se često koristi u 
nosivim biomedicinskim uređajima zbog svoje energetske učinkovitosti i sposobnosti za prijenos podataka s malom potrošnjom energije.
Mala potrošnja od velike je važnosti kod nosivih uređaja jer time mogu imati manju bateriju i biti lakši te raditi dulje vremensko razdoblje 
bez punjenja ili zamjene baterije.

- neka slikica

BLE komunikacija odvija se s pomoću servisa i karakteristika. Servisi u BLE protokolu predstavljaju skupine funkcionalnosti koje 
uređaj može pružiti ili koristiti. Svaki servis ima jednu ili više karakteristika koje predstavljaju konkretne podatke ili operacije koje 
se mogu izvršiti.

Razvijeni nosivi sustav konfiguriran je kao Generic ATTribute Profile (GATT) server te pruža jedan servis imena \texttt{BodyFluidMonitoring}. 
Servis \texttt{BodyFluidMonitoring} sastoji se od dvije karakteristike, \texttt{ReceiveCommand} za primanje naredbi i 
\texttt{SendData} za slanje izmjerenih podataka ispitnom okruženju.

Razvijeno aplikacijsko programsko sučelje za kontrolu BLE komunikacije sastoji se od dvije funkcije:
\begin{lstlisting}[label={lst:ble_api},style=CStyle,caption={Programsko sučelje za kontrolu BLE komunikacije},captionpos=b]
void ble_send_data(void *data_to_send, uint8_t data_len);
void ble_init(void (*ble_cmd_handler_callback)(uint8_t));
\end{lstlisting} 

Funkcija \texttt{ble\_init(void (*ble\_cmd\_handler\_callback)(uint8\_t))} inicijalizira BLE periferiju te 
postavlja pokazivač na funkciju koja se poziva kada  \texttt{ReceiveCommand} karakteristika primi naredbu. 
Razvijeni sustav podatke ispitnom okruženju šalje funkcijom \texttt{ble\_send\_data(void *data\_to\_send, uint8\_t data\_len)} 
kojoj se prosljeđuje pokazivač na podatke koji se šalju i duljinu podataka za slanje. 
Izvršavanjem navedene funkcije ažurira se vrijednost karakteristike \texttt{SendData}.

-slikica ovog

Poruke se šalju u obliku niza znakova u formatu \texttt{vrijeme;senzor;podatci}. Format podataka ovisi o senzoru čiji se podatci šalju te je zbog toga uveden dodatan sloj 
apstrakcije između aplikacije i BLE komunikacijskog sučelja.
\begin{lstlisting}[label={lst:ble_send_sensor_data},style=CStyle,caption={Funkcija za slanje rezultata mjerenja sa pojedinog senzora},captionpos=b]
void ble_api_send_sensor_data(void *data_structure, sensor_t sensor);
\end{lstlisting} 
Pri pozivu funkcije za slanje podataka (\ref{lst:ble_send_sensor_data}) kao parametri se zadaju 
pokazivač na podatke senzora i enumeracijski tip \texttt{sensor\_t} koji određuje o kojem senzoru je riječ.
Funkcija dalje u ovisnosti o senzoru priprema niz znakova za slanje te poziva ranije opisane funckije BLE komunikacijskog sučelja 
(\ref{lst:ble_api}).

-primjer poruke? yes, no, maybe, I don't know.. can you repeat the question

\end{document}