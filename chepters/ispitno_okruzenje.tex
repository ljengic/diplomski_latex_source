\documentclass[../diplomski_rad.tex]{subfiles}

\begin{document}

\sloppy

\justifying

Za učinkovitu primjenu ranije opisanog sustava potrebno je 
razviti ispitno okruženje koje omogućava ne samo prikupljanje podataka, 
već i njihovu analizu i interpretaciju na brz i intuitivan način.
U skladu s tim, u sklopu ovog rada razvijena je Python desktop aplikacija. 
Za izradu grafičkog sučelja korištena je biblioteka Qt i program Qt Designer pri čemu je 
naglasak stavljen na jednostavnost i intuitivnost korištenja.
Sve ikone korištene u aplikaciji besplatno su preuzete sa Icons8 web stranice \cite{ikone}.

\begin{figure}[htb]
    \centering
    \includegraphics[width=1\textwidth]{Figures/home.png} 
    \caption{Početni zaslon razvijene aplikacije}
    \label{slk:home}
\end{figure}

Aplikacija omogućava povezivanje s razvijenim uređajem putem BLE komuinikacijskog protokola, 
unos podataka o pacijentu,  
praćenje mjerenja u stvarnom vremenu te prikaz rezultata analize podataka. 
Podatci o pacijentima i mjerenja pohranjuju se u lokalnu bazu podataka. 
Pojedino mjerenje vezano je za pacijenta, čime se dobiva mogućnost praćenja pacijenata 
tijekom duljeg vremenskog razdoblja i usporedba raličitih mjerenja za istog pacijenta.
Početni zaslon prikazan je na slici \ref{slk:home}. 
S lijeve strane ekrana tokom cijelog rada aplikacije nalazi se izbornik kojim se korisnik lako prebacuje 
na željenu funcionalnost.  

U nastavku će detaljno biti opisani implementacija i korištenje razvijenog ispitnog okruženja.

\section{Povezivanje s razvijenim uređajem}

Ispitno okruženje konfigurirano je kao BLE klijent te se jednostavno spaja na razvijeni nosivi uređaj. 
Pritiskom na karticu izbornika \texttt{BLE Connection} otvara se zaslon prikazan na slici \ref{slk:ble} 
koji korisniku pruža mogućnost upravljanja BLE vezom.

\begin{figure}[htb]
    \centering
    \includegraphics[width=1\textwidth]{Figures/ble.png} 
    \caption{Zaslon za upravljanje BLE vezom}
    \label{slk:ble}
\end{figure}


Pritiskom na tipku \texttt{Scan}  ispitno okruženje započinje skeniranje dostupnih uređaja 
u blizini putem BLE protokola. 
Nakon što se skeniranje dovrši, korisniku se prikazuje popis dostupnih uređaja na sučelju aplikacije.

Korisnik zatim ima mogućnost odabira željenog uređaja. Nakon što je uređaj odabran, pritiskom na tipku \texttt{Connect} 
pokreće se postupak povezivanja uređaja. 
Aplikacija obaviještava korisnika o rezultatu pokušaja povezivanja te ako su uređaji uspješno povezani moguće je pokrenuti mjerenje.  
Za vrijeme dok su uređaji povezani na \texttt{BLE Connection} zaslonu prikazane su informacije o povezanom uređaju 
i tipka \texttt{Disconnect} kojom se korisniku daje mogućnost prekida BLE veze.

Za pokretanje mjerenja i prikupljanje mjernih podataka nužno je da aplikacija bude povezana sa razvijenim nosivim sustavom. 
Zbog toga se prije početka mjerenja vrši provjera je li aplikacija ostvarila BLE vezu sa razvijenim sustavom. 
Ako BLE veza nije ostvarena, aplikacija ne dozvoljava pokretanje mjerenja te korisnika o tome obaviještava 
skočnim prozorom prikazanim na slici \ref{slk:ble_not_connected}. Tipka \texttt{Connect} na skočnom prozoru korisnika 
vodi na karticu \texttt{BLE Connection}.

\begin{figure}[htb]
    \centering
    \includegraphics[width=0.4\textwidth]{Figures/ble_not_connected.png} 
    \caption{Obavijest da periferni uređaj nije povezan}
    \label{slk:ble_not_connected}
\end{figure}

Radi lakše kontrole BLE veze na statusnu traku u gornji desni dio zaslona dodan je bluetooth simbol, 
što je prikazano na slici \ref{slk:ble_status}. 
Ako je simbol zelen, razvijeni sustav povezan je s ispitnim okruženjem dok u slučaju crvenog simbola veza nije uspostavljena. 
Ovim pristupom korisniku se omogućava da u svakom trenutku korištenja aplikacije jednostavno može provjeriti status BLE veze. 

\begin{figure}[htb]
    \centering
    \includegraphics[width=0.4\textwidth]{Figures/ble_status.png} 
    \caption{Simbol u gorenjem desnom uglu koji opisuju status BLE veze \cite{ikone}}
    \label{slk:ble_status}
\end{figure}

\section{Postupak mjerenja}

Prvi korak pri pokretanju mjerenja je odabir između dva slučaja, mjerenje za novog pacijena ili 
mjerenje za pacijenta koji je od ranije u bazi podataka.

\begin{figure}[htb]
    \centering
    \includegraphics[width=1\textwidth]{Figures/patient.png} 
    \caption{Zaslon za unos podataka o pacijentu}
    \label{slk:patient}
\end{figure}

Ako se odabere novi pacijent, otvara se zaslon za unos podataka o pacijentu, pikazan na slici \ref{slk:patient}. 
Podatci koji su potrebni za daljnju analizu su dob, spol, težina, visina, obujmi prsa i noge te lijekovi koje pacijent upotrebljava. 
Nakon što korisnik popuni podatke pritiskom na tipku \texttt{Complete} pokreće se provjera ispravnosti unesenih podataka. 
Ukoliko neki od podataka nedostaju ili su u nepravilnom formatu, aplikacija obaviještava korisnika skočnim prozorom s 
porukom koji točno podataka nije ispravan. 
Ako mjerenje vršimo za pacijenta čiji su podatci od ranije u bazi podataka otvara se isti zaslon ali sa popunjenim podatcima 
koje korisnik po potrebi može promjeniti.  

\begin{figure}[htb]
    \centering
    \includegraphics[width=0.4\textwidth]{Figures/invalid_data.png} 
    \caption{Primjer obavijesti neispravno unesenog podatka}
    \label{slk:invalid_data}
\end{figure}

Kada su svi podatci ispravni, otvara se prozor (ref) kojim se pokreče mjerenje i na kojem 
se mogu u stvarnom vremenu pratiti rezultati mjerenja. 
Mjerenje se pokreče pritiskom na tipku \texttt{Start} čime se razvijenom sustavu šalje naredba \texttt{BLE\_CMD\_START}. 
Sustav tada kontinuirano šalje podatke o izmjerenoj bioimpedanciji ispitnom okruženju. 
Ispitno okruženje iz primjenih podataka izračunava volumene tijelesnih tekućina i u stvarnom vremenu ih osvježava na grafu. 
Mjerenje se prekida pritiskom na tipku \texttt{Stop}.

- zaslon za mjerenje gdje se live plota graf

\section{Prikaz rezultata}

gledamo mi rezultate..

- skocni prozor za odabir pacijenta

- izbor mjernja prozor

- prozor s rezultatima

\end{document}